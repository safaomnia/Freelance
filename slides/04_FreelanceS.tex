\section[Freelance Status]{Freelance Status}
\begin{frame}
	\frametitle{Freelance Status}
	There are three different types of working individuals:
	\begin{columns}
	    \hspace*{0.5cm}
        \begin{column}{.4\textwidth}
            \begin{center} \includegraphics[width=\textwidth]{figures/employee.jpg}
            Employee
            \end{center}
        \end{column}
        \hspace*{-0.25cm}
        \begin{column}{.4\textwidth}
            \begin{center} \includegraphics[width=\textwidth]{figures/worker.jpg}
            Worker
            \end{center}
        \end{column}
        \hspace*{-0.5cm}
        \begin{column}{.4\textwidth}
            \begin{center} \includegraphics[width=\textwidth]{figures/self-employer.png}
            Self-employer
            \end{center}
        \end{column}
    \end{columns}
\end{frame}
\subsection[Employer]{Employer}
\begin{frame}
	\frametitle{Employer}
	Presently, there’s no single legal test to determine if you’re an employee or a worker, but you’re likely to be an Employee if:
	\vspace{2mm}
	\fontsize{9}{12}\selectfont
	 \begin{itemize} \itemsep0.3em
	    \item Your Employer provides the work, any tools and equipment for it, and they decide how and when you do the work.
	    \item You’ll usually have a written contract.
	    \item You’re expected to do the work yourself that you’re employed to do, and may be moved to different tasks.
	    \item You’re paid a regular amount according to the hours you work, and you usually have to work a set amount of hours. You may also get extra pay for overtime and bonuses.
	 \end{itemize}
\end{frame}

\subsection[Worker]{Worker}
\begin{frame}
	\frametitle{Worker}
	- A worker is anyone who works for an employer under a contract of employment and performs the work personally.
	\\~\\
	- Workers are usually either:
	\vspace{1mm}
	\fontsize{9}{12}\selectfont
	 \begin{itemize} \itemsep0.3em
	    \item \textbf{Agency workers:} The Agency who finds you work pays your wages, or if you are a contractor, you may get work through an Agency but an Umbrella company pays your wages.
	    \item  \textbf{Casual Workers:} Hired directly by the Employer, often with a written contract, and usually paid via PAYE.
	    \item  \textbf{Freelancers and Contractors:} There are occasions when those who are self-employed for tax purposes.
	 \end{itemize}
\end{frame}

\subsection[Self-employed]{Self-employed}
\begin{frame}
	\frametitle{Self-employed}
	- You’re likely to be self-employed if:
	\vspace{1mm}
	\fontsize{9}{12}\selectfont
	 \begin{itemize} \itemsep0.3em
	    \item  You determine how and when you do the work within reason.
	    \item  You’re contracted to provide services to the Client/Employer over a certain period of time for an agreed fee and are not integrated into the company.
	    \item  You run your own business and take financial responsibility if it’s successful or not, and provide the main items of equipment.
	    \item  You may work on your own premises.
	 \end{itemize}
\end{frame}
